%-------------------------------------------------------------------------
% This file contains the title, author and abstract.
% It also contains all relevant document numbers if needed.
%-------------------------------------------------------------------------

% Partner Logo
% put the name of the logo image file found in graphics path
\DISPartnerLogo{ukaea_logo}


% Title
\DISTitle{Surrogate Modelling of\\the Tritium Breeding Ratio}

% Draft version:
% If given, adds draft version on front page, a 'DRAFT' box on top of each other page, 
% and line numbers.
%\DISVersion{0}

% Abstract - % directly after { is important for correct indentation
\DISAbstract{%
 The tritium breeding ratio (TBR) is an essential quantity for the design of
 modern and next-generation Tokamak nuclear fusion reactors. Representing the
 ratio between tritium fuel generated in breeding blankets at the boundary of
 the reactor, and fuel consumed during reactor runtime, the TBR depends in a
 complex manner on reactor geometry and material properties. We explored the
 training of surrogate models to produce a cheap but high-quality approximation
 for a Monte Carlo TBR model in use at the UK Atomic Energy Authority. We
 investigated possibilities for dimensional reduction on the parameter space of
 this model, and reviewed $N$ families of surrogate models for potential % TODO
 applicability. Here we present the performance and scaling properties of these
 surrogate models, the best of which demonstrated an accuracy of $X$ and a mean % TODO
 prediction time of $X$, representing a relative speedup $X$ with respect to the % TODO
 MC model. We further present a novel adaptive sampling algorithm, QASS, capable
 of interfacing with any of the individual studied models. Our preliminary
 testing on a toy TBR theory has demonstrated the efficacy of this algorithm for
 speeding up the surrogate modelling process.
}


% Authors and list of contributors to the analysis
\usepackage{authblk}
\author[a]{Petr Mánek}
\author[a]{Graham Van Goffrier}
\affil[a]{University College London}


% DIS reference code if ever used
%\DISRefCode{UCLCDTDIS-2019-XX}
