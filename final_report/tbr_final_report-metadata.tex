%-------------------------------------------------------------------------
% This file contains the title, author and abstract.
% It also contains all relevant document numbers if needed.
%-------------------------------------------------------------------------

% Partner Logo
% put the name of the logo image file found in graphics path
\DISPartnerLogo{ukaea_logo}


% Title
\DISTitle{Surrogate Modelling of\\the Tritium Breeding Ratio}

% Draft version:
% If given, adds draft version on front page, a 'DRAFT' box on top of each other page, 
% and line numbers.
%\DISVersion{0}

% Abstract - % directly after { is important for correct indentation
\DISAbstract{%
 The tritium breeding ratio (TBR) is an essential quantity for the design of
 modern and next-generation Tokamak nuclear fusion reactors. Representing the
 ratio between tritium fuel generated in breeding blankets and fuel consumed
 during reactor runtime, the TBR depends on reactor geometry and material
 properties in a complex manner. In this work, we explored the
 training of surrogate models to produce a cheap but high-quality approximation
 for a Monte Carlo TBR model in use at the UK Atomic Energy Authority. We
 investigated possibilities for dimensional reduction of its feature space, reviewed
 9~families of surrogate models for potential
 applicability, and performed hyperparameter optimisation. Here we present the
 performance and scaling properties of these
 models, the fastest of which, an artificial neural network,
 demonstrated~$R^2=\num{0.985}$ and a mean
 prediction time of~$\SI{0.898}{\micro\second}$, representing a relative speedup of $8\cdot 10^6$
 with respect to the expensive MC model. We further present a novel adaptive
 sampling algorithm, Quality-Adaptive Surrogate Sampling, capable
 of interfacing with any of the individually studied surrogates. Our preliminary
 testing on a toy TBR theory has demonstrated the efficacy of this algorithm for
 accelerating the surrogate modelling process.
}


% Authors and list of contributors to the analysis
\usepackage{authblk}
\author[a]{Graham Van Goffrier}
\author[a]{Petr Mánek}
\affil[a]{University College London}


% DIS reference code if ever used
%\DISRefCode{UCLCDTDIS-2019-XX}
