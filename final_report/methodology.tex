Assuming that input has been appropriately treated to eliminate redundant
features, we may turn to characterise proposed surrogate models and the criteria
used for their evaluation. The task all presented surrogates strive to solve can be
formulated using the language of conventional regression problems. In the scope
of this work, we explore various possible choices available to us in the
scheme of supervised and unsupervised learning.

Labeling the expensive Monte Carlo simulation $f(x)$, a surrogate is a mapping
$\hat{f}(x)$ that yields similar images as $f(x)$. In other words, $f(x)$ and
$\hat{f}(x)$ minimise a selected similarity metric. Furthermore, in order to
be considered \textit{viable}, surrogates are required to achieve expected evaluation time
that does not exceed the expected evaluation time of $f(x)$.

In the supervised learning setting, we first gather a sufficiently large
training set of samples $\mathcal{T}=\left\{\left( x^{(i)},f\left(x^{(i)}\right) \right)\right\}_{i=1}^N$
to describe the behaviour of $f(x)$ across its domain.
Depending on specific model class and appropriate choice of its
hyperparameters, surrogate models $\hat{f}(x)$ are trained to minimise
empirical risk with respect to $\mathcal{T}$ and a model-specific
loss function $\mathcal{L}$, where empirical risk is defined as
$R_{\text{emp.}}(\hat{f}\mid\mathcal{T},\mathcal{L})
	=\frac{1}{N}\sum_{i=1}^N
	\mathcal{L}\left(\hat{f}(x^{(i)}),f(x^{(i)})\right)$.


The unsupervised setting can be viewed as an extension of this method.
Rather than fixing the training set $\mathcal{T}$ for the entire duration of
training, multiple sets $\{\mathcal{T}_k\}_{k=0}^K$ are used, such that
$\mathcal{T}_{k-1}\subset\mathcal{T}_k$ for all $k>1$. The first set
$\mathcal{T}_0$ is initialised randomly to provide a \textit{burn-in}, and is
repeatedly extended in epochs, whereby each epoch trains a new surrogate on
$\mathcal{T}_k$ using the supervised learning procedure, evaluates its
performance, and forms a new set $\mathcal{T}_{k+1}$ by adding more samples to
$\mathcal{T}_k$. This permits the learning algorithm to condition the selection
of new samples by the results of evaluation in order to focus on improvement of
surrogate performance in complex regions within the domain.


\subsection{Metrics}
\label{sec:metrics}

Aiming to provide objective comparison of a diverse set of surrogate model
classes, we define a multitude of metrics to be tracked during experiments.
Following the motivation of this work, two desirable properties of surrogates
arise: (i) their capability to approximate the expensive
model well and (ii) their time of evaluation. An ideal surrogate would maximise
the former while minimising the latter.

\Cref{tbl:metrics} provides exhaustive listing and description of metrics recorded
in the experiments. For regression performance analysis, we include a selection
of absolute metrics to assess the approximation capability of surrogates, and set
practical bounds on the expected accuracy of their predictions. In addition, we also track
relative measures that are better-suited for model comparison between works as
they maintain invariance with respect to the selected domain and image space.
For analysis of evaluation time, surrogates are assessed in terms of wall time
elapsed during training and predicition. This closely models the practical use
case, in which they are trained and used as drop-in replacements for the
expensive model. Since training set sizes remain to be determined, all times are
reported per a single sample. Even though some surrogates support acceleration
by means of parallelisation, measures were taken to ensure sequential
processing of samples to achieve comparability between considered models.

\begin{table}[h]
	\centering
	\begin{tabular}{llrl}
	\toprule
	Regression performance metrics	& Mathematical formulation / description &
	\multicolumn{2}{c}{Ideal value} \\
	\midrule
	Mean absolute error (MAE)	& $\sum_{i=1}^N |y^{(i)}-\hat{y}^{(i)}|/N$ & 0
								& [TBR] \\
	Standard error of regression $S$	& $\text{StdDev}_{i=1}^N\left\{ |y^{(i)} -
	\hat{y}^{(i)}| \right\} $	 & 0 & [TBR] \\
	Coefficient of determination $R^2$	& $1-\sum_{i=1}^N \left(y^{(i)}-\hat{y}^{(i)} \right)^2 /
	\sum_{i=1}^N \left( y^{(i)}-\overline{y} \right)^2 $ & 1 & [rel.] \\
	Adjusted $R^2$	& $1-(1-R^2)(N-1)/(N-P-1)$	& 1 & [rel.] \\
	\midrule
	Evaluation time metrics	& {} & {} & {} \\
	\midrule
	Mean training time $\overline{t}_{\text{trn.}}$	& $(\text{wall training time of
	$\hat{f}(x)$})/N_0$ 	& 0 & [ms] \\
	Mean prediction time $\overline{t}_{\text{pred.}}$	& $(\text{wall prediction time of
	$\hat{f}(x)$})/N$	& 0 & [ms] \\
	Relative speedup	& $(\text{wall evaluation time of $f(x)$}) /
	(N\overline{t}_{\text{pred.}})$	&
	$\to\infty$ & [rel.] \\
	\bottomrule
	\end{tabular}
	\caption{Metrics recorded in supervised learning experiments. In
	formulations, we work with training set of size $N_0$ and testing set of
size $N$, TBR values $y^{(i)}=f(x^{(i)})$ and $\hat{y}^{(i)}=\hat{f}(x^{(i)})$
denote images of the $i$th testing sample in the expensive model and the surrogate
respectively. Furthermore, the mean $\overline{y}=\sum_{i=1}^N y^{(i)}/N$ and $P$ is the
number of input features.}
	\label{tbl:metrics}
\end{table}

To prevent undesirable bias in results due to training set selection, all metrics
collected in the scheme of $k$-fold cross-validation with a standard choice of
$k=5$. Herein, a sample set is subdivided into 5 disjoint folds which are
repeatedly interpreted as training and testing sets, maintaining a constant
ratio of samples between the two. In each such interpretation experiments are
repeated, and the overall value of each metric of interest is reported as the
mean across all folds.


\subsection{Supervised Learning Experiments}
\label{sec:supervised}

\subsubsection{Considered Surrogates}

\begin{table}[h]
	\centering
	\begin{tabular}{llll}
	\toprule
	Surrogate & Acronym & Implementation & Hyperparameters \\
	\midrule
	Support vector machines	& SVM & SciKit Learn & TODO \\
	Gradient boosted trees	& GBT & SciKit Learn & TODO \\
	Extremely random trees	& ERT & SciKit Learn & TODO \\
	AdaBoost	& AB & SciKit Learn & TODO \\
	Gaussian process regression	& GPR & SciKit Learn & TODO \\
	$k$ nearest neighbours	& KNN & SciKit Learn & TODO \\
	Neural networks	& NN & Keras (TensorFlow) & TODO \\
	Inverse distance weighing & IDW & SMT & TODO \\
	Radial basis functions & RBF & SMT & TODO \\
	Stochastic gradient descent & SGD & SciKit Learn & TODO \\
	Ridge regression & RR & SciKit Learn & TODO \\
	Kriging & KRG & SMT & TODO \\
	\bottomrule
	\end{tabular}
	\caption{TODO}
	\label{tbl:surrogates}
\end{table}

TODO: describe classes of surrogates

TODO: reference implementations and original papers

TODO: define NN architectures


\subsubsection{Experiments}

TODO: single slice hyperparameter optimisation

TODO: multislice (mixed) hyperparameter optimisation

TODO: retraining for scaling benchmark

TODO: retraining on large set

\subsection{Adaptive Sampling}
\label{sec:adaptive}

