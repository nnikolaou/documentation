Assuming that input has been appropriately treated to eliminate redundant
features, we may turn to characterise proposed surrogate models and the criteria
used for their evaluation. The task all presented surrogates strive to solve can be
formulated using the language of conventional regression problems. In the scope
of this work, we explore various possible choices available to us in the
scheme of supervised and unsupervised learning.

Labeling the expensive Monte Carlo simulation $f(x)$, a surrogate is a mapping
$\tilde{f}(x)$ that yields similar images as $f(x)$. In other words, $f(x)$ and
$\tilde{f}(x)$ minimise a selected similarity metric. Furthermore, in order to
be considered \textit{viable}, surrogates are required to achieve expected evaluation time
that does not exceed the expected evaluation time of $f(x)$.

In the supervised learning setting, we first gather a sufficiently large
training set of samples $\mathcal{T}=\left\{\left( x^{(i)},f\left(x^{(i)}\right) \right)\right\}_{i=1}^N$
to describe the behaviour of $f(x)$ across its domain.
Depending on specific model class and appropriate choice of its
hyperparameters, surrogate models $\tilde{f}(x)$ are trained to minimise
empirical risk with respect to $\mathcal{T}$ and a model-specific
loss function $\mathcal{L}$, where empirical risk is defined as

\begin{align}
	R_{\text{emp.}}(\tilde{f}\mid\mathcal{T},\mathcal{L})
	=\frac{1}{N}\sum_{i=1}^N
	\mathcal{L}\left(\tilde{f}(x^{(i)}),f(x^{(i)})\right).
\end{align}

The unsupervised setting can be viewed as an extension of this procedure.
Rather than fixing the training set $\mathcal{T}$ for the entire duration of
training, the points of evaluation $\{x^{(i)}\}_{i=1}^N$ that determine the set
are first initialised randomly, and continuously extended throughout training. This
permits the learning algorithm to motivate the choice of new points following the
evaluation of surrogates trained thus far by appropriately biasing the
proposal distribution, in order to better focus on problematic regions within
the domain.



\subsection{Model Comparisons}
\label{sec:model}


\subsection{Adaptive Sampling}
\label{sec:adaptive}

