\documentclass{article}

\usepackage[left=2.5cm,right=2.5cm,top=1cm,bottom=1cm]{geometry}
\usepackage[utf8]{inputenc}
\usepackage{graphicx}
\usepackage{siunitx}
\usepackage{amsmath}
\usepackage{amssymb}
\usepackage{amsthm}

\graphicspath{{fig/}}

\title{Surrogate Modelling of the Tritium Breeding Ratio}
\date{2020\\ January -- April}
\author{
	Graham Van Goffrier, University College London\\
	\and Petr Mánek, University College London
}

\begin{document}

\maketitle
\thispagestyle{empty}

The tritium breeding ratio (TBR) is an essential quantity for the design of
modern and next-generation Tokamak nuclear fusion reactors. Representing the
ratio between tritium fuel generated in breeding blankets and fuel consumed
during reactor runtime, the TBR depends on reactor geometry and material
properties in a complex manner. In this work, we explored the
training of surrogate models to produce a cheap but high-quality approximation
for a Monte Carlo TBR model in use at the UK Atomic Energy Authority.

Having implemented and deployed our sampling software on a high performance cluster, we
used the MC TBR model to generate over~$9\cdot 10^5$ datapoints for training and
test purposes. On this set, we investigated possibilities for
dimensional reduction using Principal Component Analysis, Kriging and
autoencoders, and concluded that no straightforward reduction was possible
without significant information loss. We reviewed 9~families of surrogate models for potential
applicability, benchmarked their behaviour as a function of training set size and tuned their
hyperparameters via Bayesian Optimisation.

Here we present the performance and scaling properties of these
models, the fastest of which, an artificial neural network,
demonstrated~$R^2=\num{0.985}$ and a mean
prediction time of~$\SI{0.898}{\micro\second}$, representing a relative speedup of $8\cdot 10^6$
with respect to the expensive MC TBR model. While this surrogate was trained on
a set of size $5\cdot 10^5$, we demonstrated that comparable results can also be
achieved using only $10^4$ datapoints. We further present a novel adaptive
sampling algorithm, Quality-Adaptive Surrogate Sampling, capable
of interfacing with any of the individually studied surrogates. Our preliminary
testing on a toy TBR theory has demonstrated the efficacy of this algorithm for
accelerating the surrogate modelling process.

\begin{figure}[h]
	\centering
	\includegraphics[width=0.6\textwidth]{exp4_model6}
	\caption{Regression performance of the most accurate evaluated surrogate
		(artificial neural network), viewed as true vs.~predicted TBR on a test
		set of a selected cross-validation fold (out of 5). Points are coloured by density.}
\end{figure}

\end{document}

