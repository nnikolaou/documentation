\documentclass[12pt]{iopart}

\usepackage{iopams}
\usepackage{siunitx}
\usepackage[colorinlistoftodos]{todonotes} % TODO: remove me before submitting

\providecommand{\keywords}[1]{\textbf{\textit{Index terms---}} #1}

\begin{document}
\title[Fast Regression of the Tritium Breeding Ratio in Tokamaks]{Fast Regression of the
Tritium Breeding Ratio in Tokamak Fusion Reactors} % TODO: revise this

\author{G~Van Goffrier$^1$ and P~Mánek$^{1,2}$, V~Gopakumar$^3$, N~Nikolau$^1$, J~Shimwell$^3$, I~Waldmann$^1$}

\address{$^1$ Department of Physics and Astronomy, University College London, Gower Street, London WC1E~6BT, UK}
\address{$^2$ Institute of Experimental and Applied Physics, Czech Technical University, Husova 240/5, Prague 110~00, Czech Republic}
\address{$^3$ UK Atomic Energy Authority, Culham Science Centre, OX14~3DB Abingdon, UK}

% TODO: check if Graham can also have a shorter version of email
\eads{\mailto{graham.vangoffrier.19@ucl.ac.uk}, \mailto{petr.manek@ucl.ac.uk}}

\begin{abstract}
	% TODO: this has been taken as is from the report, revise this
	The tritium breeding ratio (TBR) is an essential quantity for the design of
	modern and next-generation Tokamak nuclear fusion reactors. Representing the
	ratio between tritium fuel generated in breeding blankets and fuel consumed
	during reactor runtime, the TBR depends on reactor geometry and material
	properties in a complex manner. In this work, we explored the
	training of surrogate models to produce a cheap but high-quality approximation
	for a Monte Carlo TBR model in use at the UK Atomic Energy Authority. We
	investigated possibilities for dimensional reduction of its feature space, reviewed
	9~families of surrogate models for potential
	applicability, and performed hyperparameter optimisation. Here we present the
	performance and scaling properties of these
	models, the fastest of which, an artificial neural network,
	demonstrated~$R^2=\num{0.985}$ and a mean
	prediction time of~$\SI{0.898}{\micro\second}$, representing a relative speedup of $8\cdot 10^6$
	with respect to the expensive MC model. We further present a novel adaptive
	sampling algorithm, Quality-Adaptive Surrogate Sampling, capable
	of interfacing with any of the individually studied surrogates. Our preliminary
	testing on a toy TBR theory has demonstrated the efficacy of this algorithm for
	accelerating the surrogate modelling process.
\end{abstract}

% TODO: revise these
\keywords{magnetic moment, solar neutrinos, astrophysics}
\submitto{\jpg}
\maketitle

\todo{This is a comment that will appear in the margin}

\end{document}
