Surrogate models were developed to resolve computational limitations in the analysis of massive datasets by replacing a resource-expensive procedure with a much cheaper approximation
\cite{Sondergaard2003}. They are especially useful in applications where
numerous evaluations of an expensive procedure are required over the same or
similar domains, e.g. in the parameter optimisation of a theoretical model. The
term ``metamodel'' proves especially meaningful in this case, when the surrogate
model approximates a computational process which is itself a model for a
(perhaps unknown) physical process~\cite{Myers2002}. There exists a spectrum
between ``physical'' surrogates which are constructed with some contextual
knowledge in hand, and ``empirical'' surrogates which are derived purely from
samples of the underlying expensive model.

We sought to develop an empirical surrogate model for the tritium breeding ratio (TBR) in a Tokamak nuclear fusion reactor. Our expensive model was a Monte Carlo (MC) neutronics simulation, \textit{Paramak}\footnote{Provided by collaborator Jonathan Shimwell, at UKAEA.}, which returns a prediction of the TBR for a given configuration of a spherical Tokamak.

For the remainder of Section 1, we will define the TBR and further motivate our research. In Section 2 we will present our methodologies for the comparison testing of a wide variety of surrogate modelling techniques, as well as a novel adaptive sampling procedure suited to this application. After delivering the results of these approaches in Section 3, we will give our final conclusions and recommendations for further work.

\subsection{Problem Description}
\label{sec:problemdescription}

Nuclear fusion technology relies on the production and containment of an
extremely hot and dense plasma containing enriched Hydrogen isotopes. The current frontier generation of fusion reactors, such as the Joint European Torus (JET) and the
under-construction International Thermonuclear Experimental Reactor (ITER), make
use of both tritium and deuterium fuel. While at least one deuterium atom occurs for every \num{5000} molecules of naturally-sourced water, and may be easily distilled, tritium is extremely rare in nature. It may be produced indirectly through irradiation of heavy water
(\DDO) during nuclear fission, but only at very low rates which could
never sustain industrial-scale fusion power.

Instead, modern D-T reactors rely on tritium breeding blankets, specialised
layers of material which partially line the reactor and produce tritium upon
neutron bombardment, e.g. by 
\begin{eqnarray}
	\isotope[1][0]{n} + \hspace{3pt} \isotope[6][3]{Li} 
	&\longrightarrow \hspace{3pt} 
	\isotope[3][1]{T} + \hspace{3pt}\isotope[4][2]{He} \\
	\isotope[1][0]{n} + \hspace{3pt} \isotope[7][3]{Li} 
	&\longrightarrow \hspace{3pt} 
	\isotope[3][1]{T} + \hspace{3pt} \isotope[4][2]{He} + \hspace{3pt} \isotope[1][0]{n}
\end{eqnarray}

where T represents tritium and \isotope[7]{Li}, \isotope[6]{Li} are the more and
less common isotopes of lithium, respectively. The TBR is defined as the ratio between tritium generation in the breeding blanket and consumption in the reactor, whose input parameters in the-expensive MC model fall
into two classes. Continuous parameters, including material thicknesses and
packing ratios, describe the geometry of a given reactor configuration. Discrete
categorical parameters then specify material selections, including
coolants, armours, and neutron multipliers. We set out to produce a fast TBR
function which takes these same input parameters and approximates the MC TBR
model with the greatest achievable regression performance.

