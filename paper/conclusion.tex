\todo{This section needs to be reformulated.}

Over the course of this project, we employed a broad spectrum of data
analysis and machine learning techniques to develop fast and high-quality
surrogates for a~MC TBR simulation in use at~UKAEA. Having implemented a sampling
software to efficiently evaluate this expensive MC~model, we deployed it on a
high performance cluster to generate over~\num{900000}
datapoints for training and test purposes. We
investigated possibilities for simplification of the parameter space, and
concluded that no straightforward reduction was possible. After reviewing
9~surrogate model families, examining their behaviour on constrained and
unrestricted feature space, and studying their scaling properties, we retrained
the best-performing instances to produce properties desirable for
practical use. The fastest surrogate, an artificial neural network trained
on~\num{500000} datapoints, featured~$R^2=\num{0.985}$ with mean prediction time
of~$\SI{0.898}{\micro\second}$, representing a relative
speedup of $8\cdot 10^6$ with respect to the MC model. Alternatively, we
also demonstrated the possibility of achieving comparable results using only a
training set of size~\num{10000}.

After a thorough review of the literature, we developed a novel adaptive
sampling algorithm, QASS, capable of interfacing with any of the studied
surrogates. Preliminary testing on a toy theory, qualitatively comparable to
the MC TBR model, demonstrated the effectiveness of QASS and behavioural trends
consistent with the design of the algorithm. With~\num{100000} initial samples and 100 incremental samples per iteration, QASS achieved a ${\sim}40\%$ decrease in surrogate error as compared with a baseline random sampling scheme. Equivalently, the same surrogate error as the baseline could be achieved with ${\sim}6\%$ fewer total samples of the expensive theory. Further optimisation over the hyperparameter space has strong
potential to increase this performance, in particular by decreasing the required quantity of initial samples. This will allow for future deployment of QASS
in coalition with any of the most effective identified surrogates to facilitate
learning during evaluation of the MC TBR model.

